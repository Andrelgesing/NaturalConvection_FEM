\documentclass[12pt,a4paper]{scrartcl}
\usepackage[top=2cm, bottom=3cm, left=2.4cm, right=2.4cm]{geometry}
\usepackage{amsmath, amssymb,mathtools}\allowdisplaybreaks[1]
\usepackage{bbm}
%\usepackage{epsfig}

\usepackage[toc,page]{appendix}
\usepackage{color}
\usepackage{mathrsfs}

\usepackage{graphicx}
%\usepackage{import}
\usepackage{pgfplots}					% include tikz graphics
\usetikzlibrary{fadings}%
\usepackage{tikz}
\usetikzlibrary{arrows}
\usetikzlibrary{decorations.markings}

\usepackage{algorithm}
\usepackage[noend]{algpseudocode}
\usepackage[round]{natbib}

\usepackage{auto-pst-pdf}
\usepackage{psfrag}
\usepackage[FIGTOPCAP]{subfigure}
\usepackage{amsmath} \allowdisplaybreaks[1]
\usepackage{amsfonts}
\usepackage{tabularx}
\usepackage{xcolor}
\usepackage{verbatim}
\usepackage{rotating}
\usepackage{tikz}
\usepackage[colorlinks,citecolor=blue]{hyperref}

\addtokomafont{caption}{\small}
\setkomafont{captionlabel}{\sffamily\bfseries}
\setcapindent{0em}

%\newsavebox{\foobox}

\sloppy
\linespread{1.}
\renewcommand{\textfraction}{0.1}
\renewcommand{\topfraction}{0.9}
\renewcommand{\bottomfraction}{0.9}
\setcounter{topnumber}{3} \setcounter{bottomnumber}{3} \setcounter{totalnumber}{10}
\newcounter{lno}

\renewcommand{\vec}[1]{{\mbox{\boldmath $ #1 $}}}
\newcommand{\vecs}[1]{{\mbox{\scriptsize\boldmath $ #1 $}}}
\newcommand{\vecss}[1]{{\mbox{\tiny\boldmath $ #1 $}}}
\newcommand{\ii}{\textrm{i}}
\newcommand{\ee}{\textrm{e}}
\newcommand{\dd}{\textrm{d}}
\newcommand{\ov}[1]{\overline{ #1}}
\newcommand{\br}[1]{\left\langle {#1} \right\rangle}
\newcommand{\ddt}{\frac{\dd}{\dd t}}

\newcommand{\dV}{\ \dd V }
\newcommand{\dS}{\ \dd S }

\newcommand{\lsim}{\ \genfrac{}{}{0pt}{1}{_{_{\scriptstyle <}}}{^{^{\scriptstyle \sim}}} \ }
\newcommand{\gsim}{\ \genfrac{}{}{0pt}{1}{_{_{\scriptstyle >}}}{^{^{\scriptstyle \sim}}} \ }
\newcommand{\N}{\mathbbm{N}}
\newcommand{\Z}{\mathbbm{Z}}
\newcommand{\Q}{\mathbbm{Q}}
\newcommand{\R}{\mathbbm{R}}
\newcommand{\C}{\mathbbm{C}}
\newcommand{\I}{\mathbbm{I}}

\def\Rey{\mbox{Re}}   % Reynolds number
\def\Ri{\mbox{R}}    % Richardson number
\def\sRey{\mbox{\scriptsize Re}}   % small Reynolds number
\def\sRi{\mbox{\scriptsize R}}   % small Richardson number
\def\Nb{\mbox{N}}   % Brunt-V\"{a}is\"{a}l\"{a} frequency
\def\M{\mbox{M}}      % Mach number
\def\Pe{\mbox{Pe}}    % Peclets number
\def\Pran{\mbox{Pr}}  % Prandtl number, cf plain TeX's \Pr product
\def\Gr{\mbox{Gr}}    % Grashof number
\def\Ra{\mbox{Ra}}    % Rayleigh number
\def\Nu{\mbox{Nu}}    % Nusselt number
\def\St{\mbox{St}}    % Strouhal number
\def\Kn{\mbox{Kn}}    % Knudsen number
\def\Bo{\mbox{Bo}}    % Bond number
\def\Ta{\mbox{T}}    % Taylor number
\def\Bi{\mbox{Bi}}    % Biot number
\def\Ma{\mbox{Ma}}    % Biot number

% perturbation variables
\newcommand{\up}{\vec{u}^\prime}
\newcommand{\thetap}{\theta^\prime}
\newcommand{\pp}{p^\prime}

% base flow
\newcommand{\ub}{\vec{u}_0}

% modes variable
\newcommand{\uhat}{\vec{\hat{u}}}
\newcommand{\thetahat}{\hat{\theta}}
\newcommand{\phat}{\hat{p}}
\newcommand{\qhat}{\vec{\hat{q}}}

% adjoint perturbation variables
\newcommand{\qa}{\vec{q}^\dagger}
\newcommand{\ua}{\vec{u}^\dagger}
\newcommand{\pa}{p^\dagger}
\newcommand{\thetaa}{\theta^\dagger}

\graphicspath{{./figures/}}

\begin{tikzfadingfrompicture}[name=custom fade]%
\path(-0.2cm,0.2cm) rectangle (1.2cm,-2cm); % Arrow line is an overlay!
\pgfinterruptboundingbox
\draw[very thick,transparent!20,->] (0cm,0cm) .. controls +(0cm,-1cm) and +(0cm,1cm) .. (1cm,-2cm);
\endpgfinterruptboundingbox
\end{tikzfadingfrompicture}


\begin{document}

\begin{center}
{\Large{Project 2: Natural convection in a differentially heated square cavity}}
\end{center}
\paragraph{Due Date:} Friday 24.01.2020 or before

The aim of that project is to investigate the linear stability of the flow in a square of length L, subject to a
temperature gradient imposed at its boundary. Using the Boussinesq equation the flow
field will be computed by the mean of finite element method. After completing the partially blank codes, results will be compared with literature for different parameters.

\begin{figure}
\centering
\begin{tikzpicture}
\draw[style=help lines] (0,4) grid[step=2cm] (4,0);
\fill [path fading=custom fade,
      right color=blue!60!,
      left color=orange!50!white,
     ] (-0.4,-0.3) rectangle (4.4cm,0cm);
\fill[path fading=custom fade,
      right color=blue!60!,
      left color=orange!50!white,
     ] (-0.4,4.4) rectangle (4.0cm,4cm)node[pos=.5] {$T = 0.5-x, \vec u = \vec 0$};
\fill[path fading=custom fade,
      top color=orange!50!white,
      bottom color=orange!50!white,
     ] (0,0) rectangle (-0.4cm,4.4cm);
\fill[path fading=custom fade,
      top color=blue!60!,
      bottom color=blue!60!,
     ] (4.4,0.0) rectangle (4.0cm,4.4cm)    ;
\draw[ultra thick] (0,0) rectangle (4cm,4cm);
\draw [<->,very thick] (0,1.5) node (yaxis) [right] {$y$}
        |- (1.5,0) node (xaxis) [above] {$x$}; 
\draw[<->,very thick] (-0.6,0)  -- (-0.6,4)node (ltop)[midway, left]{$1$}; 
\draw[<->,very thick] (0,-0.6)  -- (4,-0.6)node (lbit)[midway, below]{$1$}; 
\end{tikzpicture}
\caption{Problem setup and boundary conditions}
\end{figure}


\section{Preliminary Questions}

We recall the Navier-Stokes equation, with the Oberbeck-Boussinesq approximation, written in dimensionless parameters
\begin{align}\label{eq:NS}
\frac{\partial  \vec u}{\partial t} + \left( \vec u\cdot \nabla \right)\vec u &= \Pran \left(-\nabla p + \frac{1}{\sqrt{\Ra}}\Delta \vec u + \vec e_y  T\right) \\
\nabla \cdot \vec u &= 0 \nonumber\\
\frac{\partial T}{\partial t}  + \vec u \cdot \nabla T &= \frac{1}{\sqrt{Ra}} \Delta T \nonumber
\end{align}
with the boundary conditions
\begin{align}
T &= x-1/2, \\
\vec u &= \vec 0\text{.}
\end{align}

\subsection{Linearization}
Let us define the baseflow $\vec q_0 = (\vec u_0 , p_0, T_0)^T$ which is the solution of the steady system of equation:
\begin{align}\label{eq:SteadyNS}
\left( \vec u_0\cdot \nabla \right)\vec u_0 &= \Pran \left(-\nabla p_0 + \frac{1}{\sqrt{\Ra}}\Delta \vec u_0 + \vec e_y  T_0\right) \\
\nabla \cdot \vec u_0 &= 0 \nonumber \\
 \vec u_0 \cdot \nabla T_0 &= \frac{1}{\sqrt{Ra}} \Delta T_0 \nonumber
\end{align}
Consider the total flow as sum of the baseflow and a perturbation: $\vec q = \vec q_0 + \epsilon\vec q'$ where $\epsilon \ll 1$  and $q'$ is of order 1.

\begin{itemize}
\item Insert the decomposition above in (\ref{eq:NS}) and using that $\epsilon \ll 1$, show that the linearised system of equation reads
\begin{align}\label{eq:LNS}
\frac{\partial  \vec u'}{\partial t} + \left( \vec u'\cdot \nabla \right)\vec u_0 + \left( \vec u_0\cdot \nabla \right)\vec u' &= \Pran \left(-\nabla p' + \frac{1}{\sqrt{\Ra}}\Delta \vec u' + \vec e_y  T'\right) \\
\nabla \cdot \vec u &= 0 \nonumber \\
\frac{\partial T'}{\partial t}  + \vec u_0 \cdot \nabla T' + \vec u' \cdot \nabla T_0 &= \frac{1}{\sqrt{Ra}} \Delta T' \nonumber
\end{align}
This system of equations correspond to the Jacobian of the system (\ref{eq:SteadyNS})
\item What are the boundary conditions of the linearized system? 
\item Show that the variational form of the linearized \textbf{Steady }Navier--Stokes equations reads
\begin{align}\label{eq:LNS_varf}
\int_V    \left(\vec u_0 \cdot \nabla  \right)\vec u'  \cdot \vec w   \ \dd V + \int_V \left(\vec u' \cdot \nabla  \right)\vec u_0 \cdot \vec w \ \dd V  + \frac{\Pran }{\sqrt{\Ra}}\int_V \nabla \vec u' \cdot \nabla \vec w \ \dd V & \\
 - \Pran \int_V T' \vec e_y \cdot \vec w \ \dd V - \Pran \int_V  p' \ \mathrm{div} \vec w \ \dd V  - \Pran \int_V  q \ \mathrm{div} \vec u \ \dd V & \nonumber\\
+ \int_V \left(\vec u_0 \nabla  \right)\vec T'  \cdot \theta   \ \dd V + \int_V \left(\vec u'  \nabla  \right)\vec T_0 \cdot \theta \ \dd V +\frac{1}{\sqrt{\Ra}} \int_V \nabla \vec T' \cdot \nabla \theta \ \dd V & = 0 \nonumber
\end{align}

\end{itemize}


\subsection{Newton Method}
In the UE, the non linear problem was solved using the Newton method implemented in FEniCS. Here we propose to create our own Newton method algorithm.
Recall that the Newton method reads
\begin{equation}
\vec q_0^{k+1} =  \vec q_0^{k} + \delta \vec q 
\end{equation}
where $\delta\vec q$ is the solution of the linear system of equation
\begin{equation} \label{eq:NewtonMethod_linprob}
\mathcal J(\vec q_0) \delta \vec q = -\mathcal F(\vec q_0)
\end{equation}
The Jacobian $\mathcal J$ is already known: it the linearized Navier--Stokes equations (\ref{eq:LNS_varf}) without the time dependence terms, and $\mathcal F(\vec q_0)$ is the variational formulation of the Navier--Stokes problem, \textit{i.e.} 
\begin{align}\label{eq:NS_varf}
\int_V    \left(\vec u_0 \cdot \nabla  \right)\vec u_0  \cdot \vec w   \ \dd V      + \frac{\Pran}{\sqrt{\Ra}}\int_V \nabla \vec u_0 \cdot \nabla \vec w \ \dd V & \\
- \Pran\int_V T_0 \vec e_y \cdot \vec w \ \dd V  - \Pran\int_V  p_0 \ \mathrm{div} \vec w \ \dd V   - \Pran\int_V  q \ \mathrm{div} \vec u_0 \ \dd V & \nonumber \\
+ \int_V \left(\vec u_0 \cdot \nabla  \right)\vec T_0  \cdot \theta   \ \dd V  + \frac{1}{\sqrt{\Ra}}\int_V \nabla \vec T_0 \cdot \nabla \theta \ \dd V & = 0 \nonumber
\end{align}


The general algorithm is then
\begin{enumerate}
\item solve the linear problem (\ref{eq:NewtonMethod_linprob}) to get $\delta \vec q$
\item evaluate $\epsilon_N = \vert\vert\delta \vec q \vert\vert_{L^2}$
\item increment $\vec q_0$:
\begin{equation}
  \vec q_0^{k+1} =    \vec q_0^{k} + \alpha \delta \vec q_0
\end{equation}
where $\alpha \in [0,1]$ is a relaxation parameter: if $\alpha=1$ this amount to the normal Newton method, whereas a very low but non zero value will diminish the impact the of method (lower convergence rate, but potentially more stable).
\item if $\epsilon_N< tol$ stop, otherewise go back to 1, where $tol$ is a chosen absolute tolerance, typically $10^{-6}$.
\end{enumerate}


\subsection{Linear Stability Analysis}
In this section we want, like in the UE to quantify how stable is the flow. To that end we decompose the perturbation of Eq. \ref{eq:LNS} in normal modes:
\begin{equation}
\vec q' = \sum\limits_{i=1}^\infty \hat{\vec q}_i \mathrm{e}^{\lambda t} + \mathrm{complex}\ \mathrm{conjugate}
\end{equation}
where the $\hat{\vec q}_i$ denotes the $i$th mode associated with the growth rate $\lambda_i$. Injecting this decomposition in the linearized Navier--Stokes equations gives an eigenvalue problem
\begin{equation}
\mathcal J(\vec q_0)\, \hat{\vec q}_i = - \mathbf M\, \hat{\vec q}_i
\end{equation}
where $\mathcal J(\vec q_0) \, \hat{\vec q}_i$ is the same matrix as in the first part on Newton method,
$\mathbf M$ is a mass matrix corresponding to the variable that are time dependent, \textit{i.e.} all except the pressure
\begin{equation}
-\mathbf{M}\,\hat{\vec q}_i = -\begin{pmatrix}
1 & 0 & 0 \\
0 & 0 & 0 \\
0 & 0 & 1 
\end{pmatrix}
\begin{pmatrix}
\hat{\vec u}_i \\
\hat{p}_i\\
\hat{T}
\end{pmatrix}
\end{equation}
The variational form associated to this matrix vector product is then nothing else than
\begin{equation}
-\int_V \hat{\vec u}_i \cdot \vec w + \hat{T}_i\,\theta  \  \dd V
\end{equation}

\section{Coding Aspects}
\subsection{Mesh Generation}
Using the codes given in the UE (3rd one with FE), implement the function \texttt{generate\_mesh(N)} that generates a $20 \times 20$ unitary mesh, and then refine it at  5\%, 1\%, and 0.5\% of the cavity length away from the wall. Then plot the mesh with the command \texttt{plot(mesh)}.
\subsection{Newton Method}
\begin{enumerate}
\item Complete the function called \texttt{Jacobian\_variational\_formulation} \label{task_Jacobian}. It should output the variational form of the \textbf{Linearized }Navier--Stokes equations. 
\item Complete the function called \texttt{NavierStokes}. It should out the variational form of the Navier--Stokes equations.
\item Complete the function called \texttt{solve\_newton\_step}, which call the two function implemented in the first and second steps, then solve the linear problem (\ref{eq:NewtonMethod_linprob}) and add the increment to the solution vector (eventually with a relaxation).
\item Complete the function called \texttt{solve\_newton}, which calls the function implemented in the third step, computes the $\epsilon_N$ and loops while it is larger than $10^{-6}$
\end{enumerate}

To reach high Rayleigh number values ($\Pran=0.71$, $\Ra > 10^5$) one needs to ramp up progressively: for that case you can for instance run $\Ra =[10^4,10^5,10^6,2\times10^6]$. 

\subsection{Linear Stability Analysis}
\begin{table}
\caption{Critical Rayleigh numbers for Prandtl numbers $\Pran = 0.71$,$\Pran = 0.015$ (water and mercury). } \label{tab:Critical}
\centering
\begin{tabular}{c|c|cc}
Method & mesh & $Pr$ & $\Ra_c$ \\
\hline
\begin{tabular}{@{}c@{}}Finite Difference \\ second order in time\end{tabular}  & stretched $80\times 80$ & $0.71$ & $2.101 \times 10^6$ \\
Spectral Method & $20\times 20$ & $0.71$ & $2.110 \times 10^6$\\
Spectral Method & $30\times 30$ & $0.71$ & $2.108 \times 10^6$ \\
\hline
Spectral Method & $20\times 20$ & $0.015$ & $44580$\\
Spectral Method & $30\times 30$ & $0.015$ & $40695$  
%Spectral Method & $20\times 20$ & $7.0$  & $4.9\times 10^6$\\
%Spectral Method & $30\times 30$ & $7.0$  & $5.6924\times 10^6$
\end{tabular}

\end{table}
\begin{enumerate}
\item Complete the function called \texttt{Compute\_eigenvalues}: you have to write the variational forms and one of them has already been implemented in  \ref{task_Jacobian}.
\item The stability boundary is characterised by having the largest eigenvalue having a zero real part. Using \verb|N = 20|, \verb|order = 2| as parameters, bracket the critical Rayleigh number $\Ra_c$ at which the eigenvalue is zero up to the second digit, for $\Pran = 0.71$ and $\Pran = 0.015$, and compare with the result from litterature which are summed up in the table \ref{tab:Critical}.
\item How does the stability boundary changes with increase of the polynomial order?
\item Compare with the results you had in the first project (The case with the higher Prandtl number can still have a mismatch).
\end{enumerate}
\subsection{General questions}
\begin{enumerate}

\item Is the implementation stable is the polynomial order for pressure is the same as the polynomial order for the velocity? What is the condition for numerical stability called? 
\item On which functional space do the velocity live? 
\item Which other kind of algorithm could we have used to compute the Steady flow? Give an example.
\item How is the time required by your computer evolving as the polynomial order increases?
\item Was it longer with you Finite Difference code to see whether the flow was stable/instable for a given Rayleigh number?
\item Do you think it would be possible to solve the eigenvalue problem for a 3D flow with \texttt{scipy}'s \texttt{eigs} (Do not do it!)?

\end{enumerate}




\end{document}